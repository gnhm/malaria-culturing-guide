\documentclass{article}
\usepackage{amsmath,amsfonts,amsthm,amssymb}
\usepackage{siunitx}
\usepackage{graphicx,float,wrapfig}
\usepackage{cite}

\author{Guilherme Nettesheim}
\title{Modeling bacterial growth}

\graphicspath{ {../images/} }

\begin{document}
\maketitle

\section{Observables}

Each cell can be described by several parameters, or observables, given by

\begin{itemize}
	\item Time since ``birth,'' or division, $t$
	\item The growth rate, in units of inverse time, $\alpha$
	\item Its volume, or size, $V$
	\item Its initial volume, or initial size, $V_0$
\end{itemize}

These four variables are, however, redundant, since we also have an ``equation of state'' given by:

\begin{align} \label{eqOfState}
V(t) = V_0 e^{\alpha t}
\end{align}

which comes from the observation of exponential growth for individual, as well as populations. Therefore, the state is three-dimensional, and any three of these four coordinates are sufficient to describe the state of a particular cell. We will refer to this three-dimensional space as $\Sigma$.


\section{Division probability}

We are interested in knowing when a cell divides. Division is seen as a stochastic event, the probability of which depends on the state of the cell. We will now deal with some basic definitions and identities. I have changed a lot of the symbols used in the literature.\\

The probability of dividing at a certain point $q \in \Sigma$ will be denoted $P(q)$. Generally, we will be dealing with explicit coordinates. For example, the probability of dividing at time $t$, given a growth-rate $\alpha$ and an initial volume $V_0$ will be denoted $P(t \mid \alpha, V_0)$, and so on.\\

Another way to express the same idea is through the \emph{survival function}, defined as the probability that the cell has ``not yet divided.'' This, of course, assumes an ordering, so we will really only use this for conditional probabilities with time or volume as the single coordinates. This is denoted $S(t \mid \alpha, V_0)$, for example\footnote{In \cite{Kennard2016}, this is denoted $P_0$ instead of $S$}. The survival function and probability distributions are related by:

\begin{align}
P(t \mid \alpha, V_0) &= - \dot{S}(t \mid \alpha, V_0)\\
\nonumber\\
S(t \mid \alpha, V_0) &= \int_t^\infty \, P(\tau \mid \alpha, V_0) \, d\tau 
\end{align}

We can express either of these through a \emph{hazard function}. The hazard function is defined through:\footnote{While \cite{Kennard2016} also use $h$ for hazard functions, they explicitely write $h_d$ to denote hazard function for division, which I find redundant. Also, they use a star to denote the hazard function in different coordinate systems, which we do not do.}

\begin{align}
\dot{S}(t \mid \alpha, V_0) = -h(t \mid \alpha, V_0) \cdot (t \mid \alpha, V_0)
\end{align}

and likewise for $V$ as the coordinate.\\

This can be solved by

\begin{align}
S(t \mid \alpha, V_0) = \exp \left(- \int_0^t \, h(\tau \mid \alpha, V_0) \, d\tau \right)\\
\nonumber\\
P(t \mid \alpha, V_0) = h(t \mid \alpha, V_0) \cdot \exp \left(- \int_0^t \, h(\tau \mid \alpha, V_0) \, d\tau \right) \label{eqHazard}
\end{align}

A quick clarification: Unfortunately, both the hazard function and the probability distribution are often refered to as the probability of division. What is the difference?  $P(t \mid \alpha, V_0)\, dt$ is the probability of dividing in the interval $[t, t + dt]$, and no sooner. If 100 cells are born at $t = 0$, and 50 of then divide at $t$ (others perhaps have divided earlier, or have not yet divided), then $P(t \mid \alpha, V_0) \approx 50$. If 50\% of the \emph{remaining}, undivided cells divide at $t$, then $h(t \mid \alpha, V_0) \approx 50$.

\section{Steady growth distribution}

So far we have been only describing one cell, and the idea of dividing abstractly.\\

A cell is ``born'' at time $t$, with initial volume $V_0$, and growth rate $\alpha$. It will grow, according to the equation of state (\ref{eqOfState}), and at a certain time $t_f$ (or, equivalently, volume $V_f$), will divide. The point at which it divides is stochastic, and controlled by the hazard function.\\

Division will create two new cells, however. Let us pick just one of these. What do we know about its inital parameters? It might have a different growth rate, $\alpha^\prime$, but we cannot say how it depends on its progenitor's growth rate. However, we can say confidently that its initial volume will be half of the final volume,

\begin{align}
V_0^\prime = \frac{1}{2}V_f
\end{align}

Thus, the initial volume of a daughter cell is a \emph{deterministic} function of the final volume of the mother cell, but because the final volume of the mother cell is a stochastic function of the mother cell's initial volume, \emph{the daughter cell's initial volume is a stochastic function of the mother cell's initial volume}.\\

We can imagine a long family line of growth and division, tracking only the initial size:

\begin{align}
V_0^{(1)} \to V_0^{(2)} \to \dots \to V_0^{(N)}
\end{align}

How does one initial size propagate to the next generation? Assume we have a cell with initial volume $V_0$. What is the probability that its descendants will have volume $V_0^\prime$? It is the probability that the cell will divide at $V = 2 V_0^\prime$. Thus, the propagator is $P(V = 2 V_0^\prime \mid \alpha, V_0)$.\\

Now let $\rho(V_0)$ be the probability distribution of initial volume of the $N^\text{th}$ generation. The probability distribution of initial volume of the $(N+1)^\text{th}$ will be given by: \footnote{I am having trouble elegantly describing where this ``two'' normalizing constant comes from. It's obviously ``two'' because division is a splitting into ``two.'' It is not a Jacobian, as far as I can tell. In any case, I'll come back to this \dots }

\begin{align}
\rho^\prime (V_0^\prime \mid \alpha) = 2\int_0^\infty \, P(V = 2 V_0^\prime \mid \alpha, V_0) \cdot \rho(V_0 \mid \alpha) \, d V_0
\end{align}

We now make an additional requirement: A steady state. We assume that the distribution of sizes does not change with subsequent generation, so that

\begin{align} \label{eqSteadyState}
\rho (V_0^\prime\mid \alpha) = 2\int_0^\infty \, P(V = 2 V_0^\prime \mid \alpha, V_0) \cdot \rho(V_0 \mid \alpha) \, d V_0
\end{align}

Note that the only thing that has changed is that the first $\rho$ has lost its prime, so that the two functions $\rho$ are now the same on both sides. This is a condition on $\rho$, which we will use in subsequent sections.

\section{Scaling assumptions}

We will now impose certain scaling assumptions, based on experimental results, and follow their consequences.\\

\subsection{Distribution of $V_0$ scales}

The first observation\cite{Kennard2016} is that the distribution of initial volumes of \emph{E. coli} cells varies across different media and conditions. However, when ``rescaled,'' all of these distributions have the same shape:

\begin{center}
\includegraphics[scale = 0.8]{eColiGrowthRescaling.png}
\end{center}

Now let us translate this into our mathematical language. Let $\langle V_0 \rangle_\alpha$ be the mean of $V_0$, taken over the distribution $\rho(V_0 \mid \alpha)$. Then

\begin{align}
\langle V_0 \rangle_\alpha \cdot \rho\left( \langle V_0 \rangle_\alpha \cdot x \mid  \alpha\right) = \langle V_0 \rangle_\beta \cdot \rho \left( \langle V_0 \rangle_\beta \cdot x \mid  \beta \right)
\end{align}

That is, if we take a dimensionless number $x$, multiply it by the mean of a distribution, and then rescale the distribution by the Jacobian (to keep the distribution normalized), this should give the same answer, regardless of the distribution (i.e., regardless of the growth rate, $\alpha$ or $\beta$). This rescaled function is therefore ``universal,'' and we will call it $R$:

\begin{align}
R(x) \equiv \langle V_0 \rangle_\alpha \cdot \rho\left( \langle V_0 \rangle_\alpha \cdot x \mid  \alpha\right)
\end{align}

for any choice of $\alpha$.\\

I'd now like to rewrite the above in a different format, to make the road ahead easier. Let $\sigma_c (x) = c\cdot x$ be the function which scales a variables by a constant $c$. Then the above relation can be written

\begin{align}
R = \sigma_{\langle V_0 \rangle_\alpha} \circ \rho \circ \sigma_{\langle V_0 \rangle_\alpha}
\end{align}

or, more illustratively, 

\begin{align}
R = \left( \sigma_{\langle V_0 \rangle_\alpha} \right)^\prime \cdot (\rho \circ \sigma_{\langle V_0 \rangle_\alpha})
\end{align}

where we have written the final scaling as a Jacobian, since that is its role. Inverting the above, we have

\begin{align}
\rho(V_0 \mid \alpha) = \left(\sigma^{-1}_{\langle V_0 \rangle_\alpha}\right)^{\prime} \cdot \left( R \circ \sigma_{\langle V_0 \rangle_\alpha}^{-1} \right) (V_0)
\end{align}

\subsection{Probability of division is universal}

Now let us use this identity, which is equivalent to the scaling condition on the distribution of initial volumes, and let us put it into the steady state equation (\ref{eqSteadyState}), to get

\begin{align}
\left(\sigma^{-1}_{\langle V_0 \rangle_\alpha}\right)^{\prime} \cdot \left( R \circ \sigma_{\langle V_0 \rangle_\alpha}^{-1} \right) (V_0) &= 2 \int_0^\infty\, \left(\sigma^{-1}_{\langle V_0 \rangle_\alpha}\right)^{\prime} \cdot \left( R \circ \sigma_{\langle V_0 \rangle_\alpha}^{-1} \right) (V_0^\prime) \cdot P(2V_0 \mid \alpha, V_0^\prime) \, d V_0^\prime\\
\nonumber\\
&= 2 \int_0^\infty\,  \left(\sigma^{-1}_{\langle V_0 \rangle_\alpha}\right)^{\prime} \cdot \left( R \circ \sigma_{\langle V_0 \rangle_\alpha}^{-1} \right) (V_0^\prime) \cdot P(2V_0 \mid \alpha, (\sigma_{\langle V_0 \rangle_\alpha} \circ \sigma_{\langle V_0 \rangle_\alpha}^{-1}) (V_0^\prime) ) \, d V_0^\prime\\
\nonumber\\
&= 2 \int_0^\infty\, R(x^\prime) \cdot P(2V_0 \mid \alpha, \sigma_{\langle V_0 \rangle_\alpha}  (x^\prime) ) \, d x^\prime\\
\nonumber\\
\left(\sigma^{-1}_{\langle V_0 \rangle_\alpha}\right)^{\prime} \cdot \left( R \circ \sigma_{\langle V_0 \rangle_\alpha}^{-1} \right) (V_0) &= 2 \int_0^\infty\, R(x^\prime) \cdot P(2V_0 \mid \alpha, \sigma_{\langle V_0 \rangle_\alpha}  (x^\prime) ) \, d x^\prime\\
\nonumber\\
\left( R \circ \sigma_{\langle V_0 \rangle_\alpha}^{-1} \right) (V_0) &= 2 \int_0^\infty\, R(x^\prime) \cdot \left(\sigma_{\langle V_0 \rangle_\alpha}\right)^{\prime} \cdot P(2V_0 \mid \alpha,\sigma_{\langle V_0 \rangle_\alpha}  (x^\prime) ) \, d x^\prime\\
\nonumber\\
R(x) &= 2 \int_0^\infty\, R(x^\prime) \cdot \left(\sigma_{\langle V_0 \rangle_\alpha}\right)^{\prime} \cdot P(2 \sigma_{\langle V_0 \rangle_\alpha}(x) \mid \alpha, \sigma_{\langle V_0 \rangle_\alpha}  (x^\prime) ) \, d x^\prime\\
\nonumber\\
R(x) &= 2 \int_0^\infty\, R(x^\prime) \cdot  Q(2 x \mid \alpha, x^\prime ) \, d x^\prime
\end{align}

What has this tedious exercise in change of variables brought us? The fact that the last equation has an alpha dependency on the right-hand-side only. Therefore, $Q$ does not depend on $\alpha$, and is similarly, a universal function:

\begin{align}
Q(2 x \mid \alpha, x^\prime ) \equiv Q(2 x \mid x^\prime )
\end{align}

In other words, the probability of a cell of original volume $V_0$, with growth-rate $\alpha$, dividing at volume $V$, only depends on the ratio of $V_0$ to the mean initial volume, and the ratio of $V$ to the mean initial volume.

\subsection{Constraint for hazard function}

We've now shown the requirement of steady growth and scaling has led to a simplification of the functional form of the probability of dividing. This must also propagate to a condition on the hazard function.\\

From equation (\ref{eqHazard}) and the universality of $P$, we can write:\\

\begin{align}
h(V \mid V_0) &= - \frac{d}{dV} \log \int^{\infty}_{V} \, P(V^\prime \mid \alpha, V_0) \, dV^\prime\\ 
\nonumber \\
&= - \frac{d}{dV} \log \int^{\infty}_{V} \, Q( \sigma_{\langle V_0 \rangle_\alpha}^{-1}(V^\prime) \mid \sigma_{\langle V_0 \rangle_\alpha}^{-1}(V_0) ) \, dV^\prime\\ 
\nonumber \\
&= - \frac{d}{dV} \log \, \langle V_0 \rangle_\alpha \cdot \int^{\sigma_{\langle V_0 \rangle_\alpha}^{-1}(\infty)}_{\sigma_{\langle V_0 \rangle_\alpha}^{-1}(V)} \, Q( \sigma_{\langle V_0 \rangle_\alpha}^{-1}(V^\prime) \mid \sigma_{\langle V_0 \rangle_\alpha}^{-1}(V_0) ) \, dV^\prime\\ 
\nonumber \\
&= - \frac{d}{dV} \log \int^{\infty}_{\sigma_{\langle V_0 \rangle_\alpha}^{-1}(V)} \, Q( x^\prime \mid \sigma_{\langle V_0 \rangle_\alpha}^{-1}(V_0) ) \, dx^\prime\\ 
\nonumber \\
&= - \frac{1}{\langle V_0 \rangle_\alpha} \frac{d}{dx} \left. \log \int^{\infty}_{x} \, Q( x^\prime \mid \sigma_{\langle V_0 \rangle_\alpha}^{-1}(V_0) ) \, dx^\prime \right|_{x = \sigma_{\langle V_0 \rangle_\alpha}^{-1} (V)}
\end{align}

We now note that other than the first factor, the right-hand-side, again, is a dimensionless, universal function of volumes scaled by the mean, initial volumes. That is, for

\begin{align}
\frac{1}{\langle V_0 \rangle_\alpha} f(x, x_0) \equiv -  \frac{d}{dx} \log \int^{\infty}_{x} \, Q( x^\prime \mid x_0 ) \, dx^\prime  
\end{align}

we have

\begin{align}
h(V \mid V_0) &= \frac{1}{\langle V_0 \rangle_\alpha} \cdot f\left( \sigma_{\langle V_0 \rangle_\alpha}^{-1}(V) \mid \sigma_{\langle V_0 \rangle_\alpha}^{-1}(V_0) \right)
\end{align}

\subsection{Summary}

To summarize:

\begin{itemize}
	\item The distribution of initial volumes, $\rho(V_0 \mid \alpha)$ is observed to scale.
	\item Requiring that $\rho(V_0 \mid \alpha)$ scale shows that there is a universal initial volume distribution, $R(x)$.
	\item The existence of a universal distribution $R$, along with the requirement of steady growth, shows that there is a universal probability distribution $Q(x^\prime \mid x)$, which is independent of $\alpha$. 
\end{itemize}

\section{Questions}

Factor of two in propagator

How does alpha change? Cell by cell? With time within a particular cell? I was sloppy with how I treated $\alpha$ in above sections

\bibliography{mybib}{}
\bibliographystyle{plain}
\end{document}

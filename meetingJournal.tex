\documentclass{article}
\usepackage{amsmath,amsfonts,amsthm,amssymb}
\usepackage{siunitx}
\usepackage{graphicx,float,wrapfig}
\usepackage{cite}

\author{Guilherme Nettesheim}
\title{Malaria Group Meeting Notes}
\date{}

\graphicspath{ {../images/} }

\begin{document}
\maketitle

\section{October 4, 2018}

First meeting. My first project will be to quantify \emph{Plasmodium falciparum} growth within erythrocytes. We will use the formalism in \cite{Kennard2016}, and hopefully end up with an equation.\\

When I'm back from my wedding, Viola will teach me and Kate to culture cells.\\

Counting will be easier with fluorescence. We have a few genetically-modified strains that are fluorescent, so that we might be able to count puncta. Because the fluorescence is localized on apical proteins, fluorescence might not be expressed constantly. Therefore, characterizing when fluorescence is expressed will be first step.\\

There is little about the dynamics of growth within cells, but there is a recent bioArxiv paper about it. Viola will send it to me.\\

A problem with imaging cells is that as invasions occur, the ruptured cells release toxins into the medium, and this hurts the other cells. Therefore, I will also have to create some sort of flow-chamber to control for this. The flow will have to be small so that the cells don't move much.\\

I will also need a hepatitis B booster to do blood work. I should get one ASAP.\\

We will also try to achieve parallelization within one year, October, 2019.\\

\section{November 15, 2018}

The first of biweekly meetings for team malaria (Me, Viola, Kate, Jurij, Boyko).\\

Discussed the possibility that RON11, the fluorescent malaria strain we have begun using, might not be as invasive as the lab strain. Preliminary counting by Kate suggests that this is not the case, but the numbers are low. Since excitation of fluorescence is in blue, it might be harming the cells.\\

The possibility of using different fluorescent strains, which become fluorescent at different points in the malaria life cycle, to study regulation, was brought up.\\

Using fluorescence + bright-field to record orientation... Cool.\\

The "curling" and inversion of RBCs is not as common-place as I believed... In fact, it's quite rare.\\


\bibliography{mybib}{}
\bibliographystyle{plain}
\end{document}

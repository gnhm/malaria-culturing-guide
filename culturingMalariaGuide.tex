\documentclass{article}
\usepackage[margin=0.5in]{geometry}
\usepackage{amsmath,amsfonts,amsthm,amssymb}
\usepackage{siunitx}
\usepackage{graphicx,float,wrapfig}
\usepackage{cite}

\author{Guilherme Nettesheim}
\title{Culturing malaria guide}

\graphicspath{ {../images/} }

\begin{document}
\maketitle

On October 19, 22, and 23 2018, Viola Introini showed Kate Webber and me how to culture malaria. Based on what I was taught, and on the protocol written by Teresa Tiffert, I have put together my own notes, for my own benefit.\\

\section{Preparation of culture media}

We begin by making the two media, ``wash medium'' and ``growth medium,'' if we have not yet done so. These media last for a month or more, if stored in the refrigerator. The condition of the media can be judged by their color. It is good practice to remake the media every month, or so.\\

\textbf{Wash medium ($\approx$ \SI{500}{mL}}):

\begin{itemize}
	\item To a \SI{500}{mL} bottle of RPMI-1640, add:
	\begin{itemize}
		\item \SI{18.75}{\milli\liter} of 1M HEPES
		\item \SI{5}{\milli\liter} 20\% glucose solution
		\item \SI{3}{\milli\liter} 1M NaOH solution
		\item \SI{1.25}{\milli\liter} gentamicin sulphate solution (\SI{10}{mg/mL}). Final concentration will be \SI{25}{mg/mL}.
		\item \SI{5}{\milli\liter} of 200 mM glutamine solution
		\item \SI{0.5}{\milli\liter} of 100 mM hypoxanthine solution (in 1M NaOH). Final concentration will be $\approx$ 0.1 mM.
	\end{itemize}
\end{itemize}
	

\textbf{Growth medium ($\approx$ \SI{200}{mL}}, final concentration = 0.5\%):

\begin{itemize}
	\item Weigh out \SI{1}{g} of Albumax II, and dissolve it in \SI{30}{mL} in wash medium
	\item Filter-sterilize the Albumax II solution by passing it through a syringe filter (pore size \SI{0.2}{\micro\meter}) into a sterile flask
	\item Add wash medium to the sterile flask to a final volume of \SI{200}{mL}.
\end{itemize}

\newpage
\section{Preparation of normal erythrocytes}

We use normal human red blood cells, obtained from volunteer screened donors. It is stored at \SI{4}{\celsius}, in \SIrange{50}{100}{mL} sterile aliquots, for three to four weeks.\\

Before the blood is used for parasite cultures, the white cells (leukocytes) and platelets (thrombocytes) must be removed using LymphoPrep (stored in the \SI{4}{\celsius} refrigerator).\\

\textbf{Washed blood (50\% hematocrit)}:

\begin{itemize}
	\item Centrifuge blood at 3,000RPM for 8 minutes.
	\item Remove the supernatant using the pump, and resuspend the blood in wash medium at a 1:1 ratio of erythrocytes:wash medium (can be cold or warm)
	\item Into a \SI{15}{mL} sterile centrifuge tube, transfer \SI{4}{mL} of LymphoPrep.
	\item Carefully layer $\approx$ \SI{8}{mL} of resuspended erythrocytes on top of the LymphoPrep. Pour against side of tube, from a height.
	\item Centrifuge the layered prep at 3,000RPM for 15 minute
	\item Remove the supernatant (i.e., everything except the bottom-most layer of erythrocytes) using the pump, and resuspend the cells in \SI{10}{mL} wash medium
	\item Centrifuge the resuspended erythrocytes at 3,000RPM for 8 minutes.
	\item Remove the supernatant using the pump, and resuspend in wash medium at a 1:1 ratio
	\item Store at \SI{4}{\celsius} for 7-10 days
\end{itemize}

\newpage
\section{Reconstituting frozen \emph{P. falciparum} culture stabilates}

Malaria parasites are frozen in liquid nitrogen for long-term storage. To study malaria, we must reconstitute the frozen culture.\\

\textbf{Thawing reagents}

First check that we have the three ``thawing reagents'' already made. If not, make them:

\begin{description}
	\item[Thaw 1 - ] 12\% (w/v) NaCl solution in double distilled water (\SI{12}{g} NaCl in \SI{100}{mL} DDW)
	\item[Thaw 2 - ] 1.8\% (w/v) NaCl solution in double distilled water (\SI{1.8}{g} NaCl in \SI{100}{mL} DDW)
	\item[Thaw 3 - ] 0.9\% (w/v) NaCl solution in double distilled water, plus 0.2\% glucose (\SI{0.9}{g} NaCl and \SI{0.2}{g} glucose in \SI{100}{mL} DDW)
	\item Filter all three solutions through a \SI{0.2}{\micro\meter} filter.\\
\end{description}

With the thawing reagents made, we can proceed to make

\textbf{Infected blood ($\approx$ 5\% hematocrit)}

\begin{itemize}
	\item Place all three solutions, plus growth medium, in the \SI{37}{\celsius} incubator
	\item Remove the desired vial of infected cells from the liquid nitrogen container, and place the vial in the \SI{37}{\celsius} incubator
	\item Transfer the infected cells to a \SI{50}{mL} sterile centrifuge tube, ideally one with volume markings all the way down
	\item Estimate the volume of infected cells. Add 0.2 times this volume of ``Thaw 1'' dropwise, while mixing by swirling (think Hollandaise sauce...)
	\item Allow to stand at room temperature for 5 minutes
	\item Add \SI{10}{mL} of ``Thaw 2'' dropwise, while mixing by swirling
	\item Allow to stand at room temperature for 5 minutes
	\item Pellet infected cells by centrifugation at 1,800RPM for 5 minutes
	\item Remove the supernatant, and resuspend the pelleted infected cells by adding \SI{10}{mL} of ``Thaw 3'' dropwise, while mixing by swirling
	\item Pellet infected cells by centrifugation at 1,800RPM for 5 minutes
	\item Remove the supernatant, and resuspend the pelleted infected cells in \SI{20}{mL} of wash medium
	\item Pellet infected cells by centrifugation at 1,800RPM for 5 minutes
	\item Remove the supernatant
	\item Add enough washed blood and growth medium to reach \SI{10}{mL} of 5\% Hct, and resuspend
	\item Transfer to a sterile tissue culture flask, add gas (1\% O$_2$, 3\% CO$_2$) for a minute, seal, and place in the \SI{37}{\celsius} incubator
\end{itemize}

The infected cells will now develop, and the fraction of infected cells (i.e., the parasitaemia) will increase. We can leave this solution over the weekend, and return to it on Monday.\\

\newpage
\section{Caring and diluting}

We can continue a malaria culture indefinitely by properly caring for it. To do so, it is important to:

\begin{itemize}
	\item Keep the density of erythrocytes sufficiently high, as they lyse and are consumed by malaria. We maintain it at 5\% haematocrit (Hct) by adding washed blood.
	\item Keep the fraction of infected red blood cells right; too low or too high, and the malaria culture dies. We want 2-3\% parasitaemia, though higher (5\%) can be good for experiments.
	\item Replenish the growth medium, removing harmful compounds, and replenishing the Albumax II
\end{itemize}

The haematocrit should always be around 5\%. The parasitaemia is the thing which will vary most. Too low, and the malaria culture will likely die out. Too high, and you'll have a Malthusian catastrophe, and they'll die out. Before leaving for the weekend, since you won't see your cells for about 72 hours, add fresh blood to achieve roughly 1\% parasitaemia.\\

To change the medium, which should be done regularly, we simple centrifuge, remove supernatant, and add fresh growth medium. However, in doing so, it's also easy to add washed blood as a way to increase the haematocrit and decrease the parasitaemia, if this is something you want to do.\\

Here is an example of this. We begin with a culture with parasitaemia $p_i$. We want to achieve a final parasitaemia of $p_f < p_i$ and a final $h_f$ Hct.

\begin{itemize}
	\item Transfer the culture to a clean, sterile, centrifuge tube of the appropriate volume
	\item Pellet the culture cells by centrifugation at 1,800RPM for 5 minutes.
	\item Remove supernatant
	\item Check the volume of pellet, $v_p$. Add washed blood equal to $v_w = 2 \cdot \left( \tfrac{p_i}{p_f} - 1) \right) v_p$. The factor of two comes from the fact that we store washed blood at 50\% Hct.
	\item We now add enough growth medium to reach $h_f$. Therefore, add enough for a final volume of $(v_p + \tfrac{v_w}{2})/h_f$.
	\item Resuspend the mixture of pelleted infected cells, washed blood, and growth medium
	\item Transfer to a sterile tissue culture flask, add gas (1\% O$_2$, 3\% CO$_2$) for a minute, seal, and place in the \SI{37}{\celsius} incubator
\end{itemize}

For a practical example, assume we begin with $p_i = 2\%$. We want $p_f = 1\%$ (it's Friday) and to keep the $h_f = 5\%$. Then,

\begin{itemize}
	\item Check the volume of pellet, $v_p = \SI{1}{mL}$. Add washed blood equal to $v_w = 2 v_p = \SI{2}{mL}$. 
	\item We now add enough growth medium to reach $h_f$. Therefore, add enough for a final volume of \SI{40}{mL}.
\end{itemize}

For $p_i = p_f$, there's no need to add washed blood, and one simply adds enough growth buffer to reach a final volume of $v_p/h_f$.\\

\textbf{To dispose of a culture:}

\begin{itemize}
	\item Add Distel, roughly \SI{10}{mL}
	\item Label the blood as disposal
	\item Let it sit for a day
	\item Dump down the dirty sink
\end{itemize}

\newpage
\section{Staining and quantifying}

To quickly gauge the parasitaemia and stage of the culture, we perform a stain.

\begin{itemize}
	\item Transfer a \SI{1}{mL} sample of cells from the culture into an Eppendorf Tube
	\item Spin the sample for a few seconds in the microcentrifuge
	\item Remove enough supernatant to leave a $\approx$ 1:1 ratio of supernatant and infected blood cells (i.e., a final solution of 50\% haematocrit), and resuspend
	\item Label a clean slide
	\item Using a yellow pipe tip, put \SI{20}{mL} of this culture blood near the label end of the slide
	\item Smear the blood by dragging the blood towards the label side, then quickly smearing away (look this up on Youtube if confused)
	\item Air dry immediately with a hair dryer
	\item Flood slide with methanol
	\item Wait 30 seconds
	\item Air dry immediately with a hair dryer
	\item Flood slide with Giemsa in phosphate buffer
	\item Wait 30 minutes
	\item Flood slide with RO water
	\item Air dry immediately with a hair dryer
\end{itemize}

The blood smear can now be imaged under the microscope. The immersion oil can be placed directly on the surface, without a coverslip.\\

We can now quantify the parasitaemia and stage by counting cells. Move the image to a ``randomly selected'' place, and stop. Count the number of healthy erythrocytes, the number of erythrocytes with ring-stage parasites, with schizont parasites, and with trophozoites. If a cell has both, do not double count. Count until you have 500 counted overall, to guarantee an uncertainty of $\approx 1\%$.

\newpage
\section{Synchronizing}

To keep parasites within a narrow age range, we must synchronize the culture. A synchronized culture which is allowd to ``free-wheel'' will drift apart, and within two days, will be unsynchronized.\\

\subsection{Sorbitol lysis synchronization}

Parasitized red blood cells become metabolically active roughly 20 hours after invasion. At this point, parasite proteins are inserted into the red cell membrane, allowing the active transport of extracellular substances into the cell cytoplasm. Therefore, after this stage---and only after this stage---the parasitized red blood cells are permeable to sorbitol, and, therefore, can be selectively lysed. The final washed pellet will contain only the younger, ring-forms.\\

If the goal is to see schizonts, then aim to do the experiment 26 hours after synchronization.\\

\begin{itemize}
	\item Warm the sorbitol solution, wash medium, and growth medium to \SI{37}{\celsius} in the incubator
	\item Transfer the culture to a clean, sterile, centrifuge tube of the appropriate volume
	\item Pellet the culture cells by centrifugation at 1,800RPM for 5 minutes.
	\item Remove supernatant
	\item Resuspend pellet in ten times the volume of 5\% sorbitol solution
	\item Leave in the \SI{37}{\celsius} incubator for 5 minutes
	\item Pellet cells by centrifugation at 1,800RPM for 5 minutes.
	\item Remove supernatant and resuspend cells in 20 times the volume of wash medium
	\item Pellet cells by centrifugation at 1,800RPM for 5 minutes.
	\item Check the volume of pellet (i.e., the haematocrit). Subtract this volume from \SI{2.5}{mL}, and multiply the result by two. Add this quantity of washed blood
	\item Add enough growth medium to reach \SI{50}{mL}. Resuspend the mixture of pelleted infected cells, washed blood, and growth medium
	\item Transfer to a sterile tissue culture flask, add gas (1\% O$_2$, 3\% CO$_2$) for a minute, seal, and place in the \SI{37}{\celsius} incubator
\end{itemize}

\subsection{Magnetic separation}

To separate out the schizont-stage infected erythrocytes, we use magnetic separation, since the hemazoin present in these cells are paramagnetic.\\

\begin{itemize}
	\item Set up the magnet on its metal stand in the hood
	\item Place the autoclaved column in the purple magnet
	\item Place an empty tube beneath column. Wash the column with \SI{3}{mL} of wash medium. Let the medium flow through, dropwise, with the plunger on top. Press on plunger if necessary.
	\item Dispose of flowthrough
	\item Place an empty tube beneath column. Flow blood through column. Add $\approx$ \SI{2.5}{mL} at a time, and let it flow through.
	\item Wash the column with \SI{3}{mL} of wash medium.
	\item The collected flowthrough blood can be disposed off, as above, or kept, as above.
	\item Remove column from magnet, and put it on top of a clean centrifuge tube
	\item Wash the column with \SI{3}{mL} of wash medium
	\item Centrifuge collected schizont cells at 1,500RPM for 5 minutes
	\item There should be only a tiny red dot of a pellet. Remove supernatant (leave a little bit to not suck up pellet!)
	\item Add \SI{400}{\micro\liter} of growth medium. Resuspend.
	\item Divide into two containers
	\item Add \SI{1}{\micro\liter} of washed blood into each, to achieve approximately 0.05\% Hct
	\item Put into incubator
\end{itemize}

To clean the column

\begin{itemize}
	\item Flow $\approx$ \SI{3}{mL} distilled water down the column
	\item Flow $\approx$ \SI{3}{mL} distilled water down the column, a second time
	\item Flow $\approx$ \SI{3}{mL} ethanol down the column
	\item Autoclave column
\end{itemize}

The cells will begin to erupt roughly 30 minutes hence.

\end{document}

\documentclass{article}
\usepackage[margin=0.5in]{geometry}
\usepackage{amsmath,amsfonts,amsthm,amssymb}
\usepackage{siunitx}
\usepackage{graphicx,float,wrapfig}
\usepackage{cite}

\author{Guilherme Nettesheim}
\title{Culturing malaria guide}

\graphicspath{ {../images/} }

\begin{document}
\maketitle

\section{Washing blood}

Blood, once received, should be washed before use. Which wash method to use depends on whether the blood was drawn by a volunteer phlebotomist or received from the blood bank. Regardless, it should be stored at \SI{4}{\celsius}, and used within a week.\\

The washed blood will be at \textbf{50\% h{\ae}matocrit}.\\

\textbf{Washing blood drawn by phlebotomist}:\\

This procedure is more involved, as it removes the white cells (leukocytes) and platelets (thrombocytes) found in freshly drawn blood.

\begin{itemize}
	\item If not already sedimented, centrifuge blood at 3,000RPM for 8 minutes
	\item Aspirate supernatant, and resuspend the blood in incomplete medium at a 1:1 ratio
	\item Into a \SI{15}{mL} sterile centrifuge tube, transfer \SI{4}{mL} of LymphoPrep
	\item Carefully layer $\approx$ \SI{8}{mL} of resuspended blood on top of the LymphoPrep. Pour against side of tube, from a height
	\item Centrifuge at 3,000RPM for 15 minutes
	\item Aspirate everything except the bottom-most layer of red blood cells. Make sure to get everything, even if you take some red blood cells as well
	\item Resuspend the cells in \SI{10}{mL} wash medium
	\item Centrifuge at 3,000RPM for 8 minutes
	\item Aspirate and resuspend in wash medium at a 1:1 ratio
\end{itemize}

\textbf{Washing blood from the blood bank}:\\

This procedure is simpler, as the blood has already been filtered of non-red blood cells.

\begin{itemize}
	\item If not already sedimented, centrifuge blood at 3,000RPM for 8 minutes
	\item Aspirate supernatant, and resuspend the blood in incomplete medium at a 1:1 ratio
	\item Centrifuge at 3,000RPM for 10 minutes
	\item Aspirate everything except the bottom-most layer of red blood cells. Make sure to get everything, even if you take some red blood cells as well
	\item Resuspend in wash medium at a 1:1 ratio
\end{itemize}

\newpage

\section{Freezing cultures}

Cultures can be frozen in liquid nitrogen for long-term storage, and thawed later.\\

Only ring-stage parasites can be frozen and thawed; later stages will not survive the process. Therefore, freezing should be done with a high percentage of rings, 5\% or so. This is easier to achieve if the culture has recently been synchronized.\\

This process is also very time sensitive, as the cells do not enjoy being in glycerolyte for any longer than they have to be.\\

\textbf{Freezing procedure}

\begin{itemize}
	\item Transfer culture to a sterile centrifuge tube
	\item Centrifuge at 1,800 RPM for 5 minutes
	\item Aspirate supernatant and resuspend in small volume of remaining supernatant
	\item Estimate the volume of the pellet. Add 0.5 times this volume of glycerolyte drop-wise, while mixing by swirling. This should be done slowly
	\item Let sit at room temperature for 5 minutes
	\item Add 1.5 times the original volume of the pellet drop-wise, while mixing by swirling
	\item Aliquot into a sterile freezing vial and place at \SI{-80}{\celsius} for 24 hours
	\item Transfer vials to liquid nitrogen for long-term storage. Make sure this transfer is done quickly and so that the vial does not warm.
\end{itemize}


\section{Thawing frozen cultures}

Malaria parasites are frozen in liquid nitrogen for long-term storage, and must be thawed carefully. Before starting, make sure you have read the instructions carefully, as it is a time-sensitive process.\\

It is important that the process not take longer than necessary, as the cells are unhappy in their glycerolyte solution. It is also important to not rush the drop-wise process; it should feel tedious.\\

\textbf{Thawing procedure}\\

\begin{itemize}
	\item Remove the frozen vial of infected cells from the liquid nitrogen container, and warm it in your hand until it has melted.
	\item Transfer the infected cells to a \SI{50}{mL} sterile centrifuge tube, ideally one with volume markings all the way down
	\item Estimate the volume of infected cells. Add 0.2 times this volume of 12\% NaCl drop-wise, while mixing by swirling. This should be done slowly.
	\item Allow to stand at room temperature for 5 minutes
	\item Add \SI{10}{mL} of 1.8\% NaCl drop-wise, while mixing by swirling
	\item Pellet infected cells by centrifugation at 1,800RPM for 5 minutes
	\item Remove the supernatant, and resuspend the pelleted infected cells by adding \SI{10}{mL} of 0.9\% drop-wise, while mixing by swirling
	\item Pellet infected cells by centrifugation at 1,800RPM for 5 minutes
	\item Remove the supernatant
	\item Resuspend in enough growth medium and fresh blood to achieve the desired h{\ae}matocrit and volume (usually 3\% Hct in \SI{10}{mL})
	\item Transfer to a sterile tissue culture flask, add gas (1\% O$_2$, 3\% CO$_2$) for a minute, seal, and place in the \SI{37}{\celsius} incubator
\end{itemize}

Immediately afterwards, smear to check whether the thawing process succeeded and to check the parasit{\ae}mia. There should only be rings. The culture should be allowed a few days to stabilize before being used for an experiment.

\newpage
\section{Caring for your culture}

A malaria culture can be kept indefinitely by properly caring for it. Each strain has its own favored conditions. Here, I will describe the best conditions for the 3D7 strain.\\

To properly care for a culture, the following is necessary:\\

\textbf{Maintain a healthy parasit{\ae}mia:}
If the percentage of infected cells becomes too high (more than 6\% or so), the culture can ``crash,'' or die out. Therefore, we want to keep the culture below this percentage. While for freezing or synchronization parasit{\ae}mia of 5\% or so is necessary, it is otherwise safer to stay around 1\%. Once a week it is suggested that the parasit{\ae}mia be brought down to 0.5\% or so.\\

\textbf{Keeping fresh medium:}
The parasites use up the medium, and so it must be periodically replaced. When diluting heavily, the majority of the medium will be fresh, so this is sufficient. Likewise, if the cells are below 1\%, there is no need to change the medium. Otherwise, the medium can be easily changed. Sediment the cells either by centrifuging at 1,800RPM for 5 minutes, or leaving the flask standing for some time. Remove the supernatant, and add fresh complete medium.\\

The volume fraction of red blood cells is the \textbf{h{\ae}matocrit}. For 3D7, anything between 3-5\% should work, but I find 3\% works best, and I use this exclusively.

\textbf{Adding fresh red blood cells:}
The parasites need fresh red blood cells to invade. As the parasite population increases, it will have to be split, which will involve the addition of fresh blood cells. The fresher the blood, the better. 

\textbf{Predicting parasit{\ae}mia:}
The simple rule for predicting what your culture will look like in the future is:
\begin{itemize}
	\item To find the population of rings tomorrow, take today's trophozoite and schizont population, and multiply by 5.
	\item To find the population of trophozoites tomorrow, take today's ring population
\end{itemize}

Therefore, a population of 1.2\% rings and 0.5\% trophozoites on Monday will likely be around 2.5\% rings and 1.2\% trophozoites on Tuesday.\\

\textbf{Splitting:}
To keep the parasit{\ae}mia for getting too high, the culture must be split. If we have a culture with 2\% trophozoites today, there will be 10\% rings tomorrow, which is far too high. If we wish to have, for example, 2\% rings instead, we must split the culture by five.\\

To split a culture by, say, five, we do as follows:

\begin{itemize}
	\item Dispose of all but \SI{10}{mL} $\div 5 = $ \SI{2}{mL}
	\item Add \SI{8}{mL} of 3\% h{\ae}matocrit blood:
	\begin{itemize}
		\item Since $0.03 \times \SI{8}{mL} = \SI{240}{\micro L}$, this is how much 100\% blood we need
		\item However, our blood is stored at 50\%, so we need $2 \times \SI{240}{\micro L} = \SI{480}{\micro L}$ of 50\% Hct blood
		\item Since $\SI{8}{mL} - \SI{480}{\micro L} \approx \SI{7.5}{mL}$, this is how much complete medium we need
	\end{itemize}
	\item Gas, seal, and put in incubator
\end{itemize}

\textbf{Preparing for the weekend:}
Unless you plan on coming in during the weekend, you should decrease the parasit{\ae}mia of your culture to 0.5\% on Friday. Then, make sure to tend to your culture on Monday morning, as the parasit{\ae}mia is likely high.

\newpage

\section{Smearing and staining}

To determine the parasit{\ae}mia, and the stage of the culture, we perform a blood smear, fixing a thin layer of blood cells onto a glass slide. We then stain the parasites, and image it with a 100X oil-immersion objective.

\begin{itemize}
	\item Transfer roughly \SI{100}{\micro L} sample of cells from the culture into an Eppendorf Tube
	\item Spin the sample for a few seconds in the microcentrifuge
	\item Remove roughly \SI{5}{\micro L} of the pellet with a pipette, mixing just a bit to get some supernatant as well
	\item Place this drop of blood on a clean slide, towards the opaque end
	\item Smear the blood by dragging the blood towards the label side, then quickly smearing away 
	\item Wait a few seconds until the blood dries
	\item Rinse with methanol, and air dry immediately with a hair dryer
	\item Flood slide with fresh, filtered Giemsa in phosphate buffer
	\item Wait 10 minutes
	\item Flood slide with RO water
	\item Air dry immediately with a hair dryer
\end{itemize}

The blood smear can now be imaged under the microscope. The immersion oil can be placed directly on the surface, without a coverslip.\\

We can now quantify the parasit{\ae}mia and stage by counting cells. Move the image to a ``randomly selected'' place, and stop. Count the number of healthy red blood cells, the number of cells with ring-stage parasites, with trophozoites, and with schizont parasites. If a cell has multiple parasites, count as if it only had one. Count until you have 500 or more cells.\\

It is worth noting the limitation with counting. The posterior distribution of the parasi{\ae}mia $p$ after counting a total of $N$ cells has a standard deviation $ = \sqrt{\frac{p(1-p)}{N}}$. Therefore, if one counts a parasit{\ae}mia of $p = 1\%$, the true value is likely to be anywhere between $p = 0.5\%$ and $p = 1.5\%$.

\section{Synchronizing}

To keep parasites within a narrow age range, we must synchronize the culture. A synchronized culture which is allowed to ``free-wheel'' will drift apart, and within two days, will be unsynchronized.\\

\subsection{Sorbitol lysis synchronization}

Parasitized red blood cells become metabolically active roughly 20 hours after invasion. At this point, parasite proteins are inserted into the red cell membrane, allowing the active transport of extracellular substances into the cell cytoplasm. Therefore, after this stage---and only after this stage---the parasitized red blood cells are permeable to Sorbitol, and, therefore, can be selectively lysed. The final washed pellet will contain only the younger, ring-forms.\\

If the goal is to see schizonts, then aim to do the experiment 26 hours after synchronization.\\

\begin{itemize}
	\item Warm the Sorbitol solution, wash medium, and growth medium to \SI{37}{\celsius} in the incubator
	\item Transfer the culture to a clean, sterile, centrifuge tube of the appropriate volume
	\item Pellet the culture cells by centrifugation at 1,800RPM for 5 minutes.
	\item Remove supernatant
	\item Resuspend pellet in ten times the volume of 5\% Sorbitol solution
	\item Leave in the \SI{37}{\celsius} incubator for 5 minutes
	\item Pellet cells by centrifugation at 1,800RPM for 5 minutes.
	\item Remove supernatant and resuspend cells in 20 times the volume of wash medium
	\item Pellet cells by centrifugation at 1,800RPM for 5 minutes.
	\item Check the volume of pellet (i.e., the haematocrit). Subtract this volume from \SI{2.5}{mL}, and multiply the result by two. Add this quantity of washed blood
	\item Add enough growth medium to reach \SI{50}{mL}. Resuspend the mixture of pelleted infected cells, washed blood, and growth medium
	\item Transfer to a sterile tissue culture flask, add gas (1\% O$_2$, 3\% CO$_2$) for a minute, seal, and place in the \SI{37}{\celsius} incubator
\end{itemize}

\subsection{Magnetic separation}

To separate out the schizont-stage infected erythrocytes, we use magnetic separation, since the hemazoin present in these cells are paramagnetic.\\

\begin{itemize}
	\item Set up the magnet on its metal stand in the hood
	\item Place the autoclaved column in the purple magnet
	\item Place an empty tube beneath column. Wash the column with \SI{3}{mL} of wash medium. Let the medium flow through, drop-wise, with the plunger on top. Press on plunger if necessary.
	\item Dispose of flow-through
	\item Place an empty tube beneath column. Flow blood through column. Add $\approx$ \SI{2.5}{mL} at a time, and let it flow through.
	\item Wash the column with \SI{3}{mL} of wash medium.
	\item The collected flow-through blood can be disposed off, as above, or kept, as above.
	\item Remove column from magnet, and put it on top of a clean centrifuge tube
	\item Wash the column with \SI{3}{mL} of wash medium
	\item Centrifuge collected schizont cells at 1,500RPM for 5 minutes
	\item There should be only a tiny red dot of a pellet. Remove supernatant (leave a little bit to not suck up pellet!)
	\item Add \SI{400}{\micro\liter} of growth medium. Resuspend.
	\item Divide into two containers
	\item Add \SI{1}{\micro\liter} of washed blood into each, to achieve approximately 0.05\% Hct
	\item Put into incubator
\end{itemize}

To clean the column

\begin{itemize}
	\item Flow $\approx$ \SI{3}{mL} distilled water down the column
	\item Flow $\approx$ \SI{3}{mL} distilled water down the column, a second time
	\item Flow $\approx$ \SI{3}{mL} ethanol down the column
	\item Autoclave column
\end{itemize}

The cells will begin to erupt roughly 30 minutes hence.

\section{Recipes and ingredients}

\subsection{Media}

Most of the work is done using ``complete medium,'' occasionally called ``growth medium.'' In addition, an ``incomplete medium'' (alternatively ``wash medium``) is occasionally used to wash.\\

These media last for a month or more at \SI{4}{\celsius}. Since the medium is often needed warm, a small amount of medium can be kept in the incubator for ready use.

\textbf{Incomplete medium a.k.a. wash medium ($\approx$ \SI{500}{mL}):}

\begin{itemize}
	\item To a \SI{500}{mL} bottle of RPMI-1640, add:
	\begin{itemize}
		\item \SI{18.75}{\milli\liter} of 1M HEPES
		\item \SI{5}{\milli\liter} 20\% glucose solution
		\item \SI{3}{\milli\liter} 1M NaOH solution
		\item \SI{1.25}{\milli\liter} gentamicin sulphate solution (\SI{10}{mg/mL}). Final concentration will be \SI{25}{mg/mL}.
		\item \SI{5}{\milli\liter} of 200 mM glutamine solution
		\item \SI{0.5}{\milli\liter} of 100 mM hypoxanthine solution (in 1M NaOH). Final concentration will be $\approx$ 0.1 mM.
	\end{itemize}
\end{itemize}
	
\textbf{Complete medium a.k.a. Growth medium ($\approx$ \SI{200}{mL}, final concentration = 0.5\%):}

\begin{itemize}
	\item Weigh out \SI{1}{g} of Albumax II, and dissolve it in \SI{30}{mL} in wash medium
	\item Filter-sterilize the Albumax II solution by passing it through a syringe filter (pore size \SI{0.2}{\micro\meter}) into a sterile flask
	\item Add wash medium to the sterile flask to a final volume of \SI{200}{mL}.
\end{itemize}

\subsection{Freezing and thawing}

\textbf{Glycerolyte solution:}

\begin{itemize}
	\item \SI{57}{\gram} glycerol
	\item \SI{1.6}{\gram} sodium lactate (lactic acid, sodium salt)
	\item \SI{30}{\milli\gram} KCl
	\item \SI{1.38}{\gram} Sodium dihydrogen phosphate (Sodium phosphate monobasic)
	\item pH to 6.8 with NaOH (1M or higher)
	\item Make up to \SI{100}{mL} with water
	\item Sterilize through \SI{0.2}{\micro\meter} filter
\end{itemize}

\textbf{Thawing solutions:}

Thawing involves three different solutions of salt. They will last forever, essentially.

\begin{description}
	\item[Thaw 1 - ] 12\% (w/v) NaCl solution in double distilled water (\SI{12}{g} NaCl in \SI{100}{mL} DDW)
	\item[Thaw 2 - ] 1.8\% (w/v) NaCl solution in double distilled water (\SI{1.8}{g} NaCl in \SI{100}{mL} DDW)
	\item[Thaw 3 - ] 0.9\% (w/v) NaCl solution in double distilled water, plus 0.2\% glucose (\SI{0.9}{g} NaCl and \SI{0.2}{g} glucose in \SI{100}{mL} DDW)
	\item Filter all three solutions through a \SI{0.2}{\micro\meter} filter.\\
\end{description}

\subsection{Smearing}

\textbf{10\% filtered Giemsa:}

This should be made fresh, daily

\begin{itemize}
	\item Mix roughly \SI{0.5}{mL} of pure Giemsa and \SI{9.5}{mL} of PBS into a syringe
	\item When using, pass solution through a \SI{0.2}{\micro\meter} filter directly onto glass slide
\end{itemize}

\end{document}

\documentclass{article}
\usepackage{amsmath,amsfonts,amsthm,amssymb}
\usepackage{siunitx}
\usepackage{graphicx,float,wrapfig}
\usepackage{cite}

\author{Guilherme Nettesheim}
\title{Making a PDMS flow cell}

\graphicspath{ {../images/} }

\begin{document}
\maketitle

On October 22, 2018, Cornelius Bausch showed me how to make a PDMS microfluidic flow cell. I took notes in my lab notebook, and am writing here for further reference.

\section{Mixing PDMS}

We only work in the PDMS bench, because PDMS makes a mess.\\

Begin with Sylgard 184 Silicone Elastomer Kit. This comes with a very large metal can of base, and a smaller jar of curing agent.\\

In a plastic cup, mix about 80g of base and an appropriate amount of curing agent. The standard ratio is 10:1, so 8 grams of curing agent. Cornelius prefers a harder gel, so he uses 8:1 (10 grams of curing agent). Nicola uses a softer gel, 15:1 (about 5 grams).\\

Mix together, for about 2 minutes. Mix very well. This will create many bubbles.\\

Put the cup in the vacuum chamber, and use the pump to remove bubbles. The solution will begin boiling over, at which point, close the valve. Either let it sit like this for 30 minutes, or open valve, let air back in, and vacuum again, cycling for about 15, until no bubbles remain.\\

\section{Pouring in PDMS}

Using an aluminum master, like Nicola's, clean the master by spraying with acetone, IPA/ethanol, and the nitrogen gun, in that order.\\

Pour the PDMS into the master slowly, to avoid forming bubbles.\\

The inevitable bubbles that do form can be removed by popping with a sharp piece of aluminum foil, or the vacuum.\\

Put the master into the 60C oven for 2 hours, or longer.\\

Any leftover, mixed PDMS can be covered and stored in the refrigerator for 24 hours. After that, it will harden.\\

\section{Attaching PDMS to glass}

After the PDMS has hardened in the master, remove it gently from the master.\\

If not using immediately, cover the channel side with tape, so as to prevent dust from gathering in the channels.\\

Punch holes in the channels, to pass fluid.\\

Cut the PDMS with a scalpel, so it can fit in the glass substrate (pietri dish, whatever).\\

Clean the PDMS surface with tape. Finally, cover both surfaces with tape, to keep clean.\\

Clean the glass surface with acetone and a lint-free cloth. Blow with nitrogen gun.\\

Place both the glass surface and the PDMS on the plasma cleaner tray. Make sure the surfaces which will stick together are exposed, i.e. facing up.\\

Vacuum pump for 3 minutes. Add gas (in our case, just air, but whatever) until 26 (units?). Plasma clean for roughly 30 seconds.\\

Remove tray, and gingerly press the PDMS surface onto glass, making sure not to touch it. Work quickly.\\

\end{document}
